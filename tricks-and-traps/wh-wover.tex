\section{wh与w over}
\label{sec:wh-and-wover}
将SAC数据读入内存,做些许修改,然后再写回到磁盘,这是最常见的操作。
有两个命令可以完成将数据写回磁盘的操作,即 !wh! 和 !w over!。

先说说这两者的区别,!w over! 会用内存中SAC文件的头段区和数据区
覆盖磁盘文件的头段区和数据区,而 !wh! 则只会用内存中SAC文件的
头段区覆盖磁盘文件中的头段区。

那么这两者该如何用呢?将SAC数据读入内存后,如果修改了数据,则必须使用
!w over!;如果仅修改了头段,则使用 !wh! 或者 !w over!
都可以,但是推荐使用 !wh!,原因很有两点:一方面,使用 !wh!
就已经足够完成操作;另一方面,!wh! 与 !w over! 相比,
前者只需要写入很少字节的数据,而后者则需要写入更多字节,因而前者在效率
上比后者要高很多,尤其是当文件较大时。
