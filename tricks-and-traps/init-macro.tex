\section{SAC初始化}
\label{sec:init-macro}
SAC为大多数命令选取了合适的默认值,但有些时候命令的默认值并不是用户
想要的,所以需要在SAC中执行命令并选取合适的值。如果能够在启动SAC时
让SAC自动执行这些命令就最好了。

SAC提供了这样的一种机制:当在终端启动SAC时,sac命令后若接文件名,则
SAC会将该文件当做SAC宏文件,并依次执行该宏文件中的SAC命令。

首先新建一个名为 !init.m! 的宏文件,其内容可以如下:
\begin{minted}{console}
qdp off
\end{minted}

然后,将该文件放在 !${SACAUX}! 目录中。在 !~/.bashrc! 中加入
如下别名语句:
\begin{minted}{console}
alias sac="${SACHOME}/bin/sac ${SACAUX}/init.m"
\end{minted}

重启shell之后,SAC在每次启动时会首先执行初始化宏文件 !init.m!
中的一系列SAC命令。

本例中的宏文件的文件名、宏文件所放置的路径以及宏文件的内容,都可以根据
需求自行决定。本例中,宏文件中只有一个语句,即 !qdp off!,执行
该命令会关闭快速绘图选项,即在绘图时将全部数据点都绘制出来。
