\section{数据申请}
申请地震波形数据,按照具体的需求,大致可以分为两大类,即事件波形数据
和连续波形数据。这两者的主要区别在于如何确定要申请的数据时间窗,前者
需要震相到时信息,后者则不需要。

\subsection{事件波形数据}
\begin{enumerate}
\item \textbf{筛选事件}:从地震目录中找出某个特定事件的信息,或根据条件
    筛选出需要的事件,筛选条件包括:经纬度范围、深度范围、时间范围、
    震级范围等
\item \textbf{筛选台站}:台站位置、仪器类型(宽频段、长周期或短周期)、
    分量类型(三分量或Z分量)
\item \textbf{确定起始和结束时间}:根据发震时刻和震中距信息,计算震相理论到时,
    由此确定要申请的数据时间窗
\end{enumerate}

事件波形数据又可以进一步分成两类:以事件为中心的事件波形数据(比如地震
定位、震源机制)和以台站为中心的事件波形数据(比如接收函数)。对于以
事件为中心的事件波形数据,通常先筛选地震事件,在筛选台站时,还可以加上
震中距范围和方位角范围的约束;对于以台站为中心的事件波形数据,通常先
确定要使用的台站,在筛选地震事件时,还可以加上震中距和反方位角范围的
约束。

\subsection{连续波形数据}
连续波形数据中没有地震事件的信息,因而都是以台站为中心的,比如背景噪声
相关。
\begin{enumerate}
\item \textbf{选择台站}:台站位置、仪器类型、分量类型
\item \textbf{确定时间范围}:选择合适的时间范围
\end{enumerate}
当然也可以根据地震事件的信息,确定申请数据的时间范围,所以,某种程度上,
事件波形数据属于连续波形数据的一个子集。
