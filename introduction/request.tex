\section{申请SAC}
在 \nameref{sec:history} 中已经说到,SAC协议仅允许IRIS将SAC源码包及
二进制包分发给地震学相关人员。所以想要从官方渠道获取SAC软件包,必须在
IRIS网站上申请。

SAC软件包申请地址:\url{http://ds.iris.edu/ds/nodes/dmc/forms/sac/}

申请的过程中需要注意如下几点:
\begin{itemize}
\item 认真填写个人信息,否则可能会被拒绝
\item 电子邮箱最好填写学术邮箱,一般邮箱可能会被拒绝\footnote{学术邮箱是指能证明你学术身份的邮箱,如edu结尾的邮箱。}
\item 若无学术邮箱,则需要其他信息证明你是地震学相关人员
\end{itemize}

IRIS提供了SAC最新版的源码包、Linux 64位下的二进制包和Mac OSX 64位下的
二进制包。其中,二进制包可以在相应的平台下直接使用,源代码包则需要编译
才能使用。具体申请那个包由用户的操作系统决定:
\begin{itemize}
\item Linux 64位系统可以申请源码包或Linux 64位包
\item Mac OSX 64位系统可以申请源码包或Mac 64位包
\item 其他系统,如Linux 32位、Mac OSX 32位、Cygwin,申请源码包
\end{itemize}

提交申请之后,需要人工审核,若审核通过,则IRIS会通过邮件将软件包
发送给你。一般审核时间为两到三个工作日。由于审核周期稍长,建议同时
申请64位二进制包和源码包。

需要注意,SAC协议规定了用户没有分发SAC软件包的权利。所以,请勿将SAC
软件包在网络上公开。
