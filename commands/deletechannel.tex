\SACCMD{deletechannel}
\label{cmd:deletechannel}

\SACTitle{概要}
从内存中的文件列表中删除一个或多个文件

\SACTitle{语法}
\begin{SACSTX}
D!ELETE!C!HANNEL! ALL
\end{SACSTX}
或
\begin{SACSTX}
D!ELETE!C!HANNEL! fname|fno|range [fname|fno|range ...]
\end{SACSTX}

\SACTitle{输入}
\begin{description}
\item [ALL] 删除内存中全部文件
\item [fname] 要删除文件的文件名
\item [filenumber] 要删除文件的文件号。第一个文件的文件号是1
\item [range] 要删除文件的文件号范围,范围用破折号分开,如!11-20!
\end{description}

\SACTitle{示例}
\begin{SACCode}
  dc 3 5                         // 删除第3、5个文件
  dc SO01.sz SO02.sz             // 删除这些名字的文件
  dc 11-20                       // 删除第11至20个文件
  dc 3 5 11-20 SO01.sz SO02.sz   // 删除上面的全部文件
\end{SACCode}
