\SACCMD{readsp}
\label{cmd:readsp}

\SACTitle{概要}
读取 \nameref{cmd:writesp} 和 \nameref{spe:writespe} 写的谱文件

\SACTitle{语法}
\begin{SACSTX}
R!EAD!SP [AMPH|RLIM|SPE] [filelist]
\end{SACSTX}

\SACTitle{输入}
\begin{description}
\item [RLIM] 读入实部和虚部分量
\item [AMPH] 读入振幅和相位分量
\item [SPE] 读取谱估计子程序文件,这个数据被从功率转换为振幅,相位分量设置为0
\item [filelist] SAC二进制数据文件列表
\end{description}

\SACTitle{缺省值}
\begin{SACDFT}
READSP AMPH
\end{SACDFT}

\SACTitle{说明}
\nameref{cmd:writesp} 命令将每个谱数据分量作为一个单独的文件写入磁盘,
你可以分别处理每个分量。这个命令让你能从两个分量重建谱数据,参见 \nameref{cmd:writesp}。
SPE选项允许你读取并转换由 \nameref{spe:writespe} 写出的谱文件格式。
这也使你可以使用 \nameref{cmd:mulomega} 和 \nameref{cmd:divomega} 命令。
