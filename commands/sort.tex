\SACCMD{sort}
\label{cmd:sort}

\SACTitle{概要}
根据头段变量的值对内存中的文件进行排序

\SACTitle{语法}
\begin{SACSTX}
SORT header [ASCEND|DESCEND] [header [ASCEND|DESCEND]...]
\end{SACSTX}

\SACTitle{输入}
\begin{description}
\item [header] 依据该头段变量的值进行文件排序
\item [ASCEND] 升序排列
\item [DESCEND] 降序排列
\end{description}

\SACTitle{说明}
根据给出的头段值对内存中的文件进行排序。头段变量在命令行中出现的越早,
这个变量字段就具有越高的优先权,后面的变量字段用于解决无法第一个变量
字段相同的情况。最多可以输入5个头段变量。

每个头段变量都可以跟着 !ASCEND! 或 !DESCEND! 来表明特定
字段的排序方式。若未指定,则默认为升序排列。如果使用 !sort! 命令
但未指定任何头段值,它将根据上一次执行 !sort! 命令时的头段去排序,
如果第一次调用 !sort! 但没有给出头段,则会报错。

\SACTitle{缺省值}
默认所有的字段都是升序排列

\SACTitle{BUGS}
\begin{itemize}
\item 在 !SAC_USE_DATABASE=1! 的情况下,使用该命令会导致数据中的台站信息丢失(v101.6)
\end{itemize}
