\SACCMD{load}
\label{cmd:load}

\SACTitle{概要}
导入外部命令(在SAC的linux版本中外部命令的导入需要额外的工作)

\SACTitle{语法}
\begin{SACSTX}
LOAD comname [ABBREV abbrevname]
\end{SACSTX}

\SACTitle{输入}
\begin{description}
\item [comname] 从一个共享目标文件中载入的一个外部函数名
\item [ABBREV abbrevname] 命令的简写
\end{description}

\SACTitle{说明}
该命令允许用户载入由SAC外部命令接口写的命令。该命令必须是一个储存在动态
库文件 !.so! 文件中。SAC将检查所有环境变量 !SACSOLIST! 中的
动态库文件,这个环境变量需要包含一个或多个动态库文件,每个库文件以空格分割。
到这些动态库文件的路径必须在环境变量 !LD_LIBRARY_PATH! 中指定。
如果 !SACSOLIST! 未设置,则SAC将使用由 !LD_LIBRARY_PATH! 中
指定的路径在动态库文件 !libsac.so! 中寻找。一个叫做 !libcom.so!
的库随着SAC发布。

\SACTitle{示例}
设置你的环境变量使得SAC在当前目录从库文件 !libbar.so! 中查找一个
称为 !foo! 的命令,并为foo设置别名为myfft:
\begin{SACCode}
% export SACSOLIST = "libcom.so libbar.so"
#  Add the current directory to the search path.
% export LD_LIBRARY_PATH {$LD_LIBRARY_PATH}:.
% sac
SAC> load foo abbrev myfft* load the command
SAC> read file1.z file2.z file3.z
SAC> myfft real-imag* invoke command with its arguments,
* commands must parse their own args.
SAC> psp
\end{SACCode}

如何创建一个包含你的命令的共享目标库:
\begin{SACCode}
Solaris::
cc -o libxxx.so -G extern.c foo.c bar.c
SGI::
cc -g -o libxxx.so -shared foo.c bar.c
LINUX: (gcc)::
gcc -o libxxx.so -shared extern.c foo.c bar.c sac.a
\end{SACCode}
其中sac.a可以从你得到sac的地方获得

\SACTitle{SAC发布的外部命令}
在SAC的发布版中有一个外部命令,叫做 !FLIPXY!。!FLIPXY!
将一个或多个X-Y数据文件作为输入,并调换X和Y。这个命令在 !${SACAUX}/external!
中的 !libcom.so! 中,同时还有 !FLIPXY! 的源代码作为参考。
为了导入 !FLIPXY!,!libcom.so! 必须包含在 !$SACSOLIST! 中。
