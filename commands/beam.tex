\SACCMD{beam}
\label{cmd:beam}

\SACTitle{概要}
利用内存中的全部数据文件计算射线束

\SACTitle{语法}
\begin{SACSTX}
BEAM [B!EARING! v] [V!ELOCITY! v] [REF!ERENCE! ON|OFF| lat lon [el]]
    [OFFSET REF|USER|STATION|EVENT|CASCADE] [E!C! anginc survel]
    [C!ENTER! x y z] [WR!ITE! fname]
\end{SACSTX}

\SACTitle{输入}
\begin{description}
\item [BEARING v] 方位,由北算起的度数
\item [VELOCITY v] 速度,单位为公里每秒
\item [REFERENCE lat lon el] 参考点,打开 !REFERENCE! 选项并定义
    参考点,这样其他文件的偏移量以此而定。!lat!、!lon!、
    !el! 分别代表纬度、经度、深度(下为正)
\item [REFERENCE ON|OFF] 开或关 !REFERENCE! 选项
\item [OFFSET REF] 偏移量是相对于 !REFERENCE! 选项设置的参考点的。
    这要求开启 !REFERENCE! 选项
\item [OFFSET USER] 偏移量直接从 !USER7!、!USER8! 以及
    !USER9! 中获取,分别代表纬度、经度以及海拔。这就要求所有文件的
    !USER7! 和 !USER8! 必须定义。如果设置了 !EC! 选项,
    则 !OFFSET USER! 要求 !USER9! 必须被设置
\item [OFFSET STATION] 偏移量相对于第一个台站的位置,这要求所有文件的
    !STLA!、!STLO! 必须定义
\item [OFFSET EVENT] 偏移量相对于第一个事件的位置,这要求所有文件的
    !EVLA!、!EVLO! 必须定义
\item [OFFSET CASCADE] SAC将会按照前面给出的顺序考虑决定偏移量的方法,
    并检查必要的数据是否具备。它将使用第一个满足要求的方法
\item [EC] 高程校正。!anginc!:入射角,从z轴算起,单位为度(震源
    距离越远,入射角越小);!survel!:表面介质速度(\si{\km\per\s})。
\item [CENTER] 用于计算射线束的中心台站。X为距参考台站的东向偏移;
    Y为距参考台站的北向偏移;Z为距参考台站的向上偏移,其单位为 \si{\m};
\item [WRITE fname] 将射线束写入磁盘
\end{description}

\SACTitle{缺省值}
\begin{SACDFT}
beam  b 90  v 9.0 ec 33  6.0 c  0. 0. 0. w BEAM
\end{SACDFT}

\SACTitle{说明}
!beam! 不覆盖SAC内存中的数据,因而当变换方位和速度时这一操作可以
重复执行。射线结果写入到磁盘文件中,并且每次可以写到不同的文件。这个设计
考虑到了用户的需求,即比较多次使用这一命令的不同结果,以寻找最佳射线束的
方位和速度。

\SACTitle{头段数据}
参见 \nameref{cmd:bbfk} 命令。

\SACTitle{错误消息}
!CENTER! 参数缺失偏移量,!EC! 参数需要正值
