\SACCMD{bbfk}
\label{cmd:bbfk}

\SACTitle{概要}
利用SAC内存中的所有文件计算宽频频率波数谱估计

\SACTitle{语法}
\begin{SACSTX}
BBFK [F!ILTER!] [NO!RMALIZE!] [EPS v] [MLM|PDS] [E!XP! n] [WA!WVENUMBER! v]
    [S!IZE! m n] [L!EVELS! n] [D!B!] [T!ITLE! text] [WR!ITE! [ON|OFF fname] [S!SQ! n]]
\end{SACSTX}

\SACTitle{输入}
\begin{description}
\item [FILTER] 使用最近一次 \nameref{cmd:filterdesign}命令设计的带通滤波器
\item [NORMALIZE] 用Capo方法归一化协方差矩阵,如果各信号道的振幅差别
    比较大,这是一个好方法
\item [EPS v] 调整协方差矩阵的分量值,矩阵对角线的项是(1.0+EPS)的整数倍
\item [MLM] 在高分辨率估计中使用最大似然法
\item [PDS] 不采用最大似然法的功率谱密度
\item [EXP n] 波数谱增加的幂次
\item [WAVENUMBER v] 从中采样谱估计的波数目
\item [SIZE m n] 极坐标中等值线的尺寸:!m! 是方位角方向上的采样
    点数;!n! 是在波数方向上的采样点数。!m!、!n!
    必须为偶数,而且其乘积最大限为40000
\item [LEVELS n] 等值线间隔数
\item [DB] 以分贝为单位的对数坐标图形
\item [TITLE text] 图形标题
\item [WRITE ON|OFF fname] 是否计算二维等值线数据并写入磁盘(xyz类型的
    SAC文件)。!fname! 是要写入的文件名或路径名。如果没有指定文件
    名,则默认为 !BBFK!
\item [SSQ n] 二维图的尺寸(取沿着正方形每个边的采样数据点),最大允许
    值为200
\end{description}

\SACTitle{缺省值}
\begin{SACDFT}
bbfk eps .01 pds exp 1 wvenumber 1.0 size 90 32 levels 11
    write off ssq 100
\end{SACDFT}

\SACTitle{头段数据}
分情况决定头段的信息:
\begin{itemize}
\item 若参考台站设置在 !kuser1! 中并且其对于所有文件是相同的,
    则所有文件的 !user7! 和 !user8! 都需要设置为偏移量
\item 若所有文件台站纬度 !stla! 以及台站经度 !stlo! 都设置了,
    则偏移量通过这些经纬度计算,以第一个文件作为参考台站
\item 若所有文件的 !user7! 和 !user8! 都设置了,则它们直接
    作为偏移量
\item 若所有文件的事件纬度 !evla! 以及事件经度 !evlo!
    都设置了,则他们用于计算偏移量,使用第一个台站作为参考台站
\end{itemize}

\SACTitle{输出}
polar输出立即被绘制出(不保留),square输出会写入到硬盘。
FK的峰值、反方位角以及波数将分别写入黑板变量 !BBFK_AMP!、
!BBFK_BAZIM! 以及 !BBFK_WVNBR!。

\SACTitle{错误消息}
\begin{itemize}
\item[-] 尺寸 !m! 或者 !n! 不是一个偶数
\item[-] 偏移量X、Y、Z未设置在头段变量 !user7!、!user8!、
    !user9!中
\item[-] 未找到 !filterdesign! 得到的系数数据,或者滤波器类型不是
    !BP!
\end{itemize}

\SACTitle{限制}
\begin{itemize}
\item 台站最多允许有100个
\item 极性等值线的最大尺寸是$m\times n = 40000$
\item 二维等值线输出的最大尺寸是$i = 200$
\end{itemize}
