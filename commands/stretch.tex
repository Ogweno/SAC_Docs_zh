\SACCMD{stretch}
\label{cmd:stretch}

\SACTitle{概要}
拉伸(增采样)数据,包含了一个可选的FIR滤波器

\SACTitle{语法}
\begin{SACSTX}
STRETCH n [F!ILTER! ON|OFF]
\end{SACSTX}

\SACTitle{输入}
\begin{description}
\item [n] 设置增采样因子,取值为2到7
\item [FILTER ON|OFF] 打开/关闭插值FIR滤波器选项
\end{description}

\SACTitle{缺省值}
\begin{SACDFT}
stretch 2 filter on
\end{SACDFT}

\SACTitle{说明}
此命令对数据进行拉伸,即增采样。当关闭滤波器选项时,仅仅在原始数据点之间
插入适当数目的零值。若使用了FIR滤波器,则通过对数据进行滤波可以创建一个
与原始波形相似但是采样周期更小的文件。需要注意的是,滤波器对频谱成分是
有影响的。

\SACTitle{头段变量}
npts、delta、e、depmin、depmax、depmen
