\SACCMD{picks}
\label{cmd:picks}

\SACTitle{概要}
控制时间标记的显示

\SACTitle{语法}
\begin{SACSTX}
PICKS [ON|OFF] [pick V!ERTICAL!|H!ORIZONTAL!|C!ROSS!] [W!IDTH! v] [H!EIGHT! v]
\end{SACSTX}

\SACTitle{输入}
\begin{description}
\item [ON|OFF] 打开/关闭时间标记的显示
\item [pick] SAC中与时间标记有关的头段变量名,可以取 !O!、!A!、
    !F!、!Tn!(n=0--9)
\item [VERTICAL] 在时间标记处绘制垂直线,标记名位于垂直线的右下方
\item [HORIZONTAL] 在最接近时间标记的数据点处绘制水平线,标记名位于线上或线下
\item [CROSS] 在时间标记处绘制垂直线,在最近的数据点处绘制水平线
\item [WIDTH v] 设置水平线的宽度为 !v!,其中宽度值为占图形宽度的比例
\item [HEIGHT v] 设置垂直线的高度为 !v!,其中高度值为占图形高度的比例
\end{description}

\SACTitle{缺省值}
\begin{SACDFT}
picks on width 0.1 height 0.1
\end{SACDFT}
所有时间标记的类型都是 !VERTICAL!

\SACTitle{说明}
该命令控制SAC绘图上时间标记的显示。这些时间标记标识了如震相到时、事件
发生时刻等。当打开显示选项时,每一个定义了的时间标记都会在绘图上相应时刻
处绘制一条线,并且在其旁有一个标记名。标记名是一个8字符长的头段变量。
头段变量中 !KA!、!KF!、!KO! 以及 !KTn!
分别是 !A!、!F!、!O! 和 !Tn! 的时间标记名。
如果标记名未定义,则标记名就是头段变量名。每一个时间标记
可以被显示为一条垂直线、一条水平线或一个交叉线。

\SACTitle{示例}
以交叉线显示时间标记 !T4!、!T5! 和 !T6!,并改变
交叉线的高度和宽度:
\begin{SACCode}
SAC> picks t4 c t5 c t6 c w 0.3 h 0.1
\end{SACCode}

\SACTitle{BUGS}
\begin{itemize}
\item !height! 选项对 !VERTICAL! 垂直线无效(v101.6a)
\end{itemize}
