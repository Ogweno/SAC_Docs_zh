\SACCMD{wiener}
\label{cmd:wiener}

\SACTitle{概要}
设计并应用一个自适应Wiener滤波器

\SACTitle{语法}
\begin{SACSTX}
W!IE!N!E!R [W!INDOW! pdw] [N!COEFF! n] [MU OFF|ON|v] [EPS!ILON! OFF|ON|e]
\end{SACSTX}

\SACTitle{输入}
\begin{description}
\item [WINDOW pdw] 设置滤波器设计窗口为 \nameref{subsec:pdw}
\item [NCOEFF n] 设置滤波器系数为 !n! 个
\item [MU off|on|v] 设置自适应步长参数
\item [MU Off] 设置自适应步长参数为0
\item [MU ON] 设置自适应步长为1.95/Rho(0),其中Rho(0)是 !pdw! 中
    延迟为0时的自相关系数
\item [MU v] 设置自适应步长为v
\item [EPSILON ON|OFF|e] 设置岭回归(ridge regression)参数为。
    若为 !OFF!,则SAC将依次设置epsilon值为0.0、$10^{-5}$、$10^{-4}$、
    $10^{-3}$、$10^{-2}$,直到滤波器稳定为止。若epsilon=0不稳定,则SAC会给出警告
    信息,若所有值均不稳定,则SAC会给出错误信息。
\end{description}

\SACTitle{缺省值}
\begin{SACDFT}
wiener window b 0 10 ncoeff 30 mu off epsilon off
\end{SACDFT}

\SACTitle{说明}
对文件的 !pdw! 内的数据估计自相关函数,并利用Yule-Walker方法设计
预测误差滤波器,然后将滤波器应用到整个信号,即信号被误差序列替换。
该滤波器可以用作预白化或用作瞬时信号的检测预处理器。若mu指定为非0值,则
滤波器为时域自适应的,大值mu可能会导致不稳定。

岭回归(ridge regression )参数通过给自相关矩阵的对角线元素加上epsilon
以达到稳定wiener滤波器的目的。

\SACTitle{示例}
下面的命令将应用一个非自适应滤波器,将第一个十秒指定为数据窗:
\begin{SACCode}
SAC> wiener window b 0 10 mu 0.
\end{SACCode}

下面命令将应用带40个系数的滤波器,指定设计窗为从文件开始到第一个到时前1秒:
\begin{SACCode}
SAC> wiener ncoeff 40 window b a -1
\end{SACCode}

\SACTitle{头段变量}
depmin、depmax、depmen
