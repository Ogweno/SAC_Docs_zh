\SACCMD{help}
\label{cmd:help}

\SACTitle{概要}
在终端显示SAC命令的语法和功能信息

\SACTitle{语法}
\begin{SACSTX}
H!ELP! [item ... ]
\end{SACSTX}

\SACTitle{输入}
\begin{description}
\item [item] 命令(全称或简写)、模块、子程序等等。若item为空,则显示
    SAC的帮助文档的介绍
\end{description}

\SACTitle{说明}
SAC的官方帮助文档位于 !$SACHOME/aux/help! 目录中,该命令实际上是从
该目录中读取相应的文件并输出到终端中。item列表中每一项会按照顺序依次显示
在终端中,若输出超过一屏,可以使用PgUp、PgDn、Enter、空格、方向键等实现
翻页。直接输入 !q! 则退出当前item的文档并显示下一item的文档。

\SACTitle{示例}
\begin{SACCode}
SAC> h                  // 获得帮助文档包的介绍
SAC> h r cut bd p       // 一次显示多个命令的文档
\end{SACCode}
