\SACCMD{plabel}
\label{cmd:plabel}

\SACTitle{概要}
定义通用标签及其属性

\SACTitle{语法}
\begin{SACSTX}
PLABEL [n] [ON|OFF|text] [S!IZE! T!INY!|S!MALL!|M!EDIUM!|L!ARGE!]
    [B!ELOW!|P!OSITION! x y [a]]
\end{SACSTX}

\SACTitle{输入}
\begin{description}
\item [n] 设置第 !n! 个通用标签的属性,若省略,则在前一通用标签
    号上加1
\item [text] 设置标签的内容,并打开通用标签选项
\item [ON] 打开通用标签选项,但不改变标签文本
\item [OFF] 关闭绘图标签选项
\item [SIZE TINY|SMALL|MEDIUM|LARGE] 修改通用标签的文本尺寸。!TINY!、
    !SMALL!、!MEDIUM!、!LARGE! 分别表示一行132、
    100、80、50个字符
\item [BELOW] 将此标签放在前一标签的下面
\item [POSITION x y a] 定义该标签的位置。其中 !x! 的取值为0到1,
    !y! 的取值为0到最大视口(一般为0.75),!a! 是标签相对
    于水平方向顺时针旋转的角度
\end{description}

\SACTitle{缺省值}
默认字体大小为 !small!,标签1的位置为 !0.15 0.2 0.!。
默认其他标签的位置为上一个标签之下。

\SACTitle{说明}
该命令允许你为接下来的绘图命令定义通用的绘图标签。你可以定义每个标签的
位置及文本尺寸。文本质量以及字体可以用 \nameref{cmd:gtext} 命令设定,
也可以使用 \nameref{cmd:title}、\nameref{cmd:xlabel}、\nameref{cmd:ylabel}
生成图形的标题以及轴标签。

\SACTitle{示例}
为绘图定义一系列标签:
\begin{SACCode}
// 三行标签
SAC> dg sub local cdv.z
SAC> plabel 'Sample seismogram' p .12 .5
SAC> plabel 'from earthquake'
SAC> plabel 'in Livermore Valley, CA'
// 放在左下角的标签
SAC> plabel 5 'LLNL station: CDV' S T P .12 .12
SAC> p
\end{SACCode}

\SACTitle{BUGS}
\begin{itemize}
\item !text! 必须用引号括起来(v101.6a)
\item !text! 的最后一个字符会被忽略(v101.6a)
\item !text! 不能以 !on! 或 !off! 开头,会被误解释
    为选项 !on! 或 !off!,前者导致直接出现段错误,后者导致
    不显示标签(v101.6a)
\end{itemize}
