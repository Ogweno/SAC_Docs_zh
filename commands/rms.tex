\SACCMD{rms}
\label{cmd:rms}

\SACTitle{概要}
计算测量时间窗内信号的均方根

\SACTitle{语法}
\begin{SACSTX}
RMS [NOISE ON|OFF|pdw] [TO USERn]
\end{SACSTX}

\SACTitle{输入}
\begin{description}
\item [NOISE ON/OFF] 打开/关闭噪声校正选项
\item [NOISE pdw] 打开噪声校正选项并设置噪声的测量时间窗 \nameref{subsec:pdw}
\item [TO USERn] 将计算结果保存到头段变量 !USERn! 中(n取0到9)
\end{description}

\SACTitle{缺省值}
\begin{SACDFT}
rms noise off to user0
\end{SACDFT}

\SACTitle{说明}
该命令用于计算当前测量时间窗(由 \nameref{cmd:mtw} 定义)内数据的
均方根,并将计算结果保存到头段变量 !USERn! 中。

均方根的定义为:
\[
 RMS = \sqrt{\frac{1}{N} \sum_{i=1}^N y_i^2}
\]

!NOISE! 选项用于校正计算结果中噪声的贡献。其计算公式为:
\[
    RMS_{corrected} = \sqrt{\frac{1}{N} \sum_{i=1}^N y_i^2 -
            \frac{1}{M} \sum_{j=1}^M y_j^2}
\]

即,先计算 !mtw! 定义的信号时间窗内数据点的平方和均值,然后计算
!NOISE pdw! 定义的噪声时间窗内数据点的平方和均值,从信号的平方
和均值中减去噪声的平方和均值,以校正噪声的贡献,最后将校正后的平方和均值
的开方作为测量结果保存到头段变量中。

\SACTitle{示例}
计算头段变量 !T1! 和 !T2! 之间的数据的未做噪声校正的均方根,
并将结果保存在头段 !USER4! 中:
\begin{SACCode}
SAC> mtw t1 t2
SAC> rms to user4
\end{SACCode}

将 !T3! 前5秒作为噪声窗,计算校正噪声后的均方根:
\begin{SACCode}
SAC> mtw t1 t2
SAC> rms noise t3 -5.0 0.0
\end{SACCode}

\SACTitle{头段变量}
!USERn!
