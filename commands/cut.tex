\SACCMD{cut}
\label{cmd:cut}

\SACTitle{概要}
定义要读入的数据窗

\SACTitle{语法}
\begin{SACSTX}
CUT [ON|OFF|pdw|SIGNAL]
\end{SACSTX}

\SACTitle{输入}
\begin{description}
\item [pdw] 打开截窗选项并修改 \nameref{subsec:pdw}
\item [ON] 打开截窗选项但不改变 !pdw!
\item [OFF] 关闭截窗选项
\item [SIGNAL] 等效于设置 !pdw! 为 !A -1 F 1!,即 !a!
    前一秒到 !f! 后一秒的数据窗
\end{description}

\SACTitle{缺省值}
\begin{SACDFT}
cut off
\end{SACDFT}

\SACTitle{说明}
!cut! 命令仅设置了要读取的时间窗选项,并不对内存中的数据进行截取。
因而,若要该命令起作用,需要在 !cut! 命令设置时间窗后使用
!read! 命令。与此相反,\nameref{cmd:cutim} 命令会在命令执行时直接
对内存中的数据进行截取。

若截窗选项为关,则读取整个文件;若截窗选项为开,则只读取由 !pdw!
定义的部分。

如果你想对一组有不同参考时刻的文件使用同样的时间窗,必须在执行 !cut!
前先使用 \nameref{cmd:synchronize} 命令使所有文件具有相同的参考时刻。
!synchronize! 命令修改了文件的头段使得所有文件具有相同的参考时刻,
并调整所有相对时间。因而,你需要先读取所有文件,执行 !synchronize!
命令,使用 \nameref{cmd:writehdr} 将修改后的头段写入到磁盘文件中,然后
再执行 !cut! 命令,并读取数据,这样才能得到正确的结果。

\SACTitle{示例}
下面将用一系列示例来展示 !cut! 命令的常见用法。首先,先生成测试用
的示例数据:
\begin{SACCode}
SAC> fg seis
SAC> w seismo.sac
\end{SACCode}
直接读取文件,不做任何截窗操作:
\begin{SACCode}
SAC> r seismo.sac
SAC> lh b e a kztime npts
          b = 9.459999e+00
          e = 1.945000e+01
          a = 1.046400e+01
     kztime = 10:38:14.000
       npts = 1000
\end{SACCode}
截取b到e之间的波形,等效于不做任何截窗操作:
\begin{SACCode}
SAC> cut b e
SAC> r seismo.sac
SAC> lh b e a kztime npts
          b = 9.459999e+00
          e = 1.945000e+01
          a = 1.046400e+01
     kztime = 10:38:14.000
       npts = 1000
\end{SACCode}
截取文件的前3秒:
\begin{SACCode}
SAC> cut b 0 3
SAC> r seismo.sac
SAC> lh b e a kztime npts
          b = 9.459999e+00
          e = 1.246000e+01
          a = 1.046400e+01
     kztime = 10:38:14.000
       npts = 301
\end{SACCode}
截取文件开始的100个数据点:
\begin{SACCode}
SAC> cut b n 100
SAC> r
SAC> lh b e a kztime npts
          b = 9.459999e+00
          e = 1.045000e+01
          a = 1.046400e+01
     kztime = 10:38:14.000
       npts = 100
\end{SACCode}
截取初动前0.5秒到初动后3秒的数据:
\begin{SACCode}
SAC> cut a -0.5 3
SAC> r
SAC> lh b e a kztime npts
          b = 9.959999e+00
          e = 1.346000e+01
          a = 1.046400e+01
     kztime = 10:38:14.000
       npts = 351
\end{SACCode}
截取数据的第10到15秒(相对于参考时刻):
\begin{SACCode}
SAC> cut 10 15
SAC> r ./seismo.sac
SAC> lh b e a kztime npts
          b = 9.999999e+00
          e = 1.500000e+01
          a = 1.046400e+01
     kztime = 10:38:14.000
       npts = 501
\end{SACCode}

先截取数据的最开始前3秒,再截取接下来的3秒:
\begin{SACCode}
SAC> cut b 0 3
SAC> r ./seismo.sac
SAC> w tmp.1
SAC> cut b 3 6
SAC> r
SAC> w tmp.2
SAC> cut off
SAC> r ./tmp.?
./tmp.1 ...tmp.2
SAC> lh b e a kztime npts

  FILE: ./tmp.1 - 1
 -------------
          b = 9.459999e+00
          e = 1.246000e+01
          a = 1.046400e+01
     kztime = 10:38:14.000
       npts = 301

  FILE: ./tmp.2 - 2
 -------------
          b = 1.246000e+01
          e = 1.546000e+01
          a = 1.046400e+01
     kztime = 10:38:14.000
       npts = 301
\end{SACCode}

当要截取的窗超过了文件的时间范围时,可以使用 \nameref{cmd:cuterr} 命令的
!FILLZ! 选项,在文件的开始或结尾处补0,再读入内存。
\begin{SACCode}
SAC> r N11A.lhz
SAC> lh npts
    npts = 3101

SAC> cuterr fillz; cut b n 4096
SAC> r
SAC> lh npts
    npts = 4096
\end{SACCode}

\SACTitle{限制}
目前不支持非等间隔文件或谱文件的截断。该命令对ASCII格式的SAC文件无效。
